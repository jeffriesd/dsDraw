\documentclass{article}

\newcommand{\bs}{\textbackslash}

\title{\vspace{-4em}\textbf{dsDraw Technical Documentation}}

\date{}
\begin{document}
\maketitle

\section*{Canvas}

The drawing tools in dsDraw are implemented primarily through the HTML5 $<$canvas$>$
element. Some features such as editing text are simplified by bringing up other
HTML elements, but once editing is done, everything on the canvas is represented
by hand-coded classes that maintain their own states. The basic structure that
is used to interface between user and canvas is the CanvasState class. 

This is the current design for CanvasState:
\\\textbf{Fields:}
\begin{itemize}
\item mouseDown: object with x, y values assigned new coordinates whenever the
user depresses the left mouse button
\item mouseMove: object with x, y values assigned new coordinates whenever the
user moves the mouse to a new coordinate (no click necessarily)
\item mouseUp: object with x, y values assigned new coordinates whenever the
user release the left mouse button
\item drawMode: string representing current drawing tool
\item canvas: reference to main HTML canvas element
\item ctx: CanvasRenderingContext2D object with actual draw/fill/line methods
\item objects: array of {type: str, canvasObj: object} objects currently on the canvas
\item ctx: CanvasRenderingContext2D object with actual draw/fill/line methods
\item clickedBare: boolean representing whether last click was on an object or bare canvas
\item activeObj: reference to most recently clicked object
\item dragOffset: x, y object representing where a drag event began (updated as object is dragged
and thus separate information from mouseDown)
\item hitCanvas: reference to hidden HTML canvas element for event detection
\item hitCtx: CanvasRenderingContext2D object for hitCanvas
\item uniqueColors: Set object with unique colors currently on hitCanvas
\item colorHash: {rgbString: canvasObject} object/mapping from colors to current objects
\item hotkeys: {keyCode: bool} object/mapping for keeping track of currently depressed hotkeys/modifiers
\end{itemize}
~\\\textbf{Methods:}
\begin{itemize}
\item setMode(string mode): change drawing mode, update toolbar label
\item undo(): remove most recently drawn object from CanvasState.objects
\item initToolbars(): initialize toolbars for different drawing modes
\item bindKeys(): initialize key bindings for hotkeys
\item addCanvasObj(objType, canvasObj): add a new object (and its type as a string) to current list 
and do color hashing
\item getClickedObject(mouseX, mouseY): returns topmost object at given coordinates
\item repaint(): clear canvas and redraw each current object (and possibly "creator" elements, such
as a hollow box if user is currently creating a new text box, etc.)
\end{itemize}


\subsection*{Canvas events}
A hidden canvas with the same dimensions as the main canvas is used for 
constant time event detection. When a new element is added to the canvas, it is
assigned a unique RGB color that is used to fill the space of the element on the hidden
canvas. When the canvas is clicked, the hidden canvas is queried for the color
at that coordinate, and the color is used as a key in a map of current canvas objects.
Using this method eliminates the need to iterate through the current canvas objects,
doing math on each to calculate its bounds and so on.

~\\TODO:
\begin{itemize}
\item Implement reordering of canvas objects so the most recently clicked objects
are drawn last (these constitute the topmost layer, so their colored hit-detection
shapes shouldn't be colored over by other elements).

\end{itemize}

\subsection*{Canvas Objects}
Canvas objects such as text boxes, arcs, arrays, etc. are represented as classes 
with a reference to the canvas state and several methods such as draw, drag, click,
release, etc.. When a click event happens, it is first determined whether it is a 
drag, release, click (initial press down), and then the corresponding method
gets called on the active canvas object.


\section{Text Objects}
Text is drawn using the HTML5 canvas fillText method and fonts are maintained by updating
the context font when drawing text. Configurable attributes are font family, size, 
style (italic, bold, etc.), alignment (horizontal and vertical), and fill/stroke color.

\end{document}
